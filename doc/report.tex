\documentclass{article} % For LaTeX2e
\usepackage{nips, times}
\usepackage{hyperref}
\usepackage{url}

\usepackage{graphicx}
\usepackage{amsmath}
\usepackage{amssymb}
\usepackage{booktabs}
\usepackage{tabularx}
\usepackage{caption}
\usepackage{subcaption}
\usepackage{color}
\usepackage{relsize}
\usepackage{placeins}
\usepackage{natbib}

\nipsfinalcopy

\begin{document}
\title{Compressed Sensing for Traffic Networks}

\author{
Cathy Wu\\
University of California\\
Berkeley, CA 94720, USA\\
\texttt{cathywu@eecs.berkeley.edu}\\
\And
Philipp Moritz\\
University of California\\
Berkeley, CA 94720, USA\\
\texttt{pcm@eecs.berkeley.edu}\\
\And
Richard Shin\\
University of California\\
Berkeley, CA 94720, USA\\
\texttt{ricshin@eecs.berkeley.edu}\\
\And
Fanny Yang\\
University of California\\
Berkeley, CA 94720, USA\\
\texttt{fanny-yang@eecs.berkeley.edu} \\
}

\maketitle

\begin{abstract}
Reconstructing true parameters from a partially observed model is a fundamental problem in science and technology.
In the last years there has been a surge of interest in this task under a broad variety of model assumptions.
Motivated by applications in traffic reconstruction, we investigate when blocks of probability distributions can be recovered from linear measurements.
We empirically study the performance of several reconstruction methods and review the literature on provably correct methods for the reconstruction of probability measures.
We evaluate several possible regularizers that can be used if the usual $\ell^1$ reconstruction is not applicable, because probability distributions have a built in $\ell^1$ constraint.
\end{abstract}

\section{Introduction}
\paragraph{Related work.}
\section{Problem formulation}
\section{Theory}
\section{Algorithms for reconstructing blocks of probability measures}
\section{Numerical Results}
\section{References}
\end{document}
